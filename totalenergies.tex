\documentclass[12pt]{article}
\usepackage[T1]{fontenc}
\usepackage[utf8]{inputenc}
\usepackage[table]{xcolor}
\usepackage{hyperref}
\usepackage{mathtools}
\usepackage{graphicx} 
\usepackage{titlesec}
\usepackage{float}
\usepackage{tabularx,booktabs,caption,ragged2e}
\usepackage[style=verbose]{biblatex}
\usepackage{eurosym}
\usepackage{amsmath , amsfonts, amssymb}
\usepackage{adjustbox}
\usepackage{fancyhdr}
\usepackage{multicol}
\addbibresource{bibliographie.bib}
\setlength{\parindent}{0pt}
\setlength{\headheight}{14.61858pt}
\captionsetup{skip=0.333\baselineskip}
\usepackage{geometry}
\geometry{a4paper,textwidth=480pt}

\begin{document}
\pagestyle{fancy}
\fancyhead{}
\fancyhead[L]{\leftmark}
\fancyhead[R]{TotalEnergies}
\fancyfoot{} 
\fancyfoot[R]{Page \thepage}

\section{Présentation du groupe TotalEnergies}
\subsection{L'histoire du groupe}
\subsection{Une multinationale}

\section{Analyse de la performance économique et financière}
\subsection{Analyse de l'activité et de la profitabilité de Total Energies}
\subsubsection{Chiffre d'affaire}
\subsubsection{Résultat opérationnel courant}
\subsubsection{Résultat opérationnel}
\subsubsection{Résultat net des entreprises intégrées}

\subsection{Analyse de la rentabilité}
\subsubsection{Rentabilité économique}
\subsubsection{Rentabilité financière}

\section{Analyse de la solvabilité}
\subsection{Analyse de la structure financière}
\subsection{Analyse de la liquidité}

\newpage
\section{Analyse des flux de trésorerie}
\subsection{Les flux de trésorerie d'exploitation}
Le tableau~\ref{table:fluxActivite} représente les flux de trésorerie d'exploitation, c'est-à-dire, les encaissements générés par les activités propres de l'entreprise comme la vente de biens et de services, et auxquels sont retirés les décaissements liés à l'exploitation.
\begin{table}[H]
    \sffamily
    \centering
    \caption{Flux de trésorerie d'exploitation}
    \label{table:fluxActivite}
    \begin{tabular}{l*{1}{rrrrrr}}
        \toprule
        \textit{(en millions de dollars)} & \textbf{2021} & 2020 & 2019 & Var $n-1$ & Var $n-2$ \\
        \midrule
        Total Energies & 30 410 & 14 803 & 24 685 & 105.43\% & 23.19\% \\ 
        \midrule
        BP & 23 612 & 12 162 & 25 770 & 94.15\% & -8.37\% \\ 
        Exxonmobil & 48 129 & 14 668 & 29 716 & 228.12\% & 61.96\% \\ 
        Shell & 45 104 & 34 105 & 42 178 & 32.25\% & 6.94\% \\ 
        Chevron & 29 187 & 10 577 & 27 314 & 175.95\% & 6.86\% \\ 
    \midrule
        Moyenne secteur & ~ & ~ & ~ & 127.18\% & 18.12\% \\
    \bottomrule
    \end{tabular}
\end{table}
Ici, pour les trois années, les flux sont positifs. Cela signifie que l'entreprise arrive à générer plus de fonds que ce qu'elle n'en dépense. Elle se sert par exemple de cet excédent pour verser des dividendes aux actionnaires. Outre une augmentation de plus de $100\%$ en 2021 par rapport à 2020 (année ayant généré de moindres flux en raison de la crise sanitaire), on remarque que l'entreprise arrive à dégager $+23,19\%$ de trésorerie due à son exploitation par rapport à l'année 2019. Ainsi, Total se démarque par rapport à d'autres entreprises du secteur, n'ayant pas réussi à générer autant de trésorerie en plus par rapport à 2019.
\subsection{Les flux de trésorerie d'investissement}
Dans le tableau~\ref{table:fluxInvest} sont représentés les flux de trésoreries d'investissement de Total, ainsi que ceux d'autres entreprises du secteur. Ce flux sera impacté négativement par les  investissements de l'entreprise (immobilisations, acquisitions de sociétés ou de titres, etc.). Au contraire, il sera impacté positivement par les désinvestissements de l'entreprise (cession d'actifs, titres etc.).
\begin{table}[H]
    \centering
    \sffamily
    \caption{Flux de trésorerie d'investissement}
    \label{table:fluxInvest}
    \begin{tabular}{l*{1}{cccccc}}
    \toprule
        \textit{(en millions de dollars)} & 2021 & 2020 & 2019 & Var n-1 & Var n-2 \\ 
    \midrule
        Total Energies & -13 656 & -13 079 & -17 177 & -4.41\% & 20.50\% \\
    \midrule
        BP & -18079 & 3 956 & -8 817 & -557.00\% & -105.05\% \\ 
        Exxonmobil & -10 235 & -18 459 & -23 084 & 44.55\% & 55.66\% \\ 
        Shell & -4 761 & -13 278 & -15 779 & 64.14\% & 69.83\% \\ 
        Chevron & -5 865 & -6 965 & -11 458 & 15.79\% & 48.81\% \\
    \midrule 
        Moyenne secteur & ~ & ~ & ~ & -87.38\% & 17.95\% \\
    \bottomrule
    \end{tabular}
\end{table}
En 2021, pour Total, le flux est négatif, cela veut dire que l'entreprise investit plus que ce qu'elle ne désinvestit, en effet en 2021 elle aura investi 16 589 millions et désinvesti 2 933 millions. Cela semble être une bonne chose pour le long terme, la politique d'investissement de Total étant de consacrer 50\% de ses investissements au maintien de ses activités historiques (pétrole) et 50\% dans la croissance de ses exploitation de gaz, électricité et énergies renouvelables, afin d'atteindre la neutralité carbone.
Par rapport à 2020, l'entreprise a légèrement fait diminuer ce flux (-4.41\%), car elle à davantage investi, si l'on compare cette évolution aux autres entreprises du secteur, Total se démarque de la majorité qui ont pour leur part vu ce flux augmenter entre 2020 et 2021, signe de désinvestissements plus importants du côté de ses entreprises.
Par rapport à 2019, le flux d'investissement a augmenté (+20.50\%), 2019 ayant été marquée par le rachat d'actifs de Anadarko en Afrique (pour 3900 M). Mis à part ce rachat, les investissements corporels et incorporels n'ont fait qu' augmenter entre 2019 et 2021 (donc diminuer le flux d'investissement)


\subsection{Les flux de trésorerie de financement}

\section{Analyse des éléments extra-financiers}
\subsection{Analyse de la gouvernance de l'entreprise}
\subsection{Analyse de la politique RSE}



\end{document}